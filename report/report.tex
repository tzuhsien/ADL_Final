%
% Homework Details
% - Title
% - Due date
% - University
% - Class
% - Class Alias
% - Class Section
% - Instructor
% - Author
% - AuthorID
%

\newcommand{\hmwkID}{1}
\newcommand{\hmwkTitle}{Final Project Report}
\newcommand{\hmwkTopic}{Game Playing with DQfD and DQN}
\newcommand{\hmwkUniversity}{NTU}
\newcommand{\hmwkClass}{Applied Deep Learning}
\newcommand{\hmwkClassAlias}{ADL}
\newcommand{\hmwkClassSection}{Fall 2017}
\newcommand{\hmwkClassInstructor}{Yun-Nung (Vivian) Chen and Hung-Yi Lee}
\newcommand{\hmwkTeam}{Praise The Sun}
\newcommand{\langver}{Trad. Chinese}



\documentclass{article}

%
% Packages
%

\usepackage{enumitem}
\usepackage{amsmath}
\usepackage{fancyhdr}
\usepackage{lastpage}
\usepackage{titlesec}
\usepackage[hidelinks]{hyperref}
\usepackage{cleveref}
\usepackage{listings}
\usepackage{bm}
\usepackage{float}
\usepackage{amsthm}
\usepackage{booktabs}

\usepackage{pgfplots}
\pgfplotsset{compat=1.7}
\usepackage{pgf,tikz}
\usepackage{mathrsfs}
\usetikzlibrary{arrows}
\usetikzlibrary[patterns]

\usepackage{xcolor}
\hypersetup{
    colorlinks,
    linkcolor={red!50!black},
    citecolor={blue!50!black},
    urlcolor={blue!80!black}
}

% additional packages
\usepackage{lettrine} % large initial
\usepackage{amsfonts} % mathbb

%
% Chinese
%

\usepackage{xeCJK}
\setCJKmainfont{Noto Sans CJK TC Regular}

\XeTeXlinebreaklocale "zh"
\XeTeXlinebreakskip = 0pt plus 1pt

%
% Basic Document Settings
%

\topmargin=-0.45in
\evensidemargin=0in
\oddsidemargin=0in
\textwidth=6.5in
\textheight=9.0in
\headsep=0.25in

\linespread{1.3}

% figure name
\renewcommand{\figurename}{圖}
\renewcommand{\tablename}{表}

\pagestyle{fancy}
%\lhead{\hmwkTeam}
\lhead{\hmwkClass\ (\hmwkUniversity, \hmwkClassSection)}
%\chead{}
\rhead{\hmwkTeam: \hmwkTitle\ (\langver)}
%\rhead{\hmwkClass\ (\hmwkUniversity, \hmwkClassSection): \hmwkTitle (\langver)}
\cfoot{\thepage\ of \pageref{LastPage}}

%
% Title Page
%

%\title{
%  % \vspace{2in}
%  \textmd{\textbf{\hmwkClass:\ \hmwkTitle}}\\
%  \textmd{\textbf{\hmwkTopic}} \\
%  \vspace{0.1in}\large{\textit{Instructors \hmwkClassInstructor}}
%  % \vspace{3in}
%}

\title{
   \vspace{-1.5cm}
  \hmwkClass:\ \hmwkTitle\\
  \textbf{\hmwkTopic} \\
  \vspace{0.1in}\large{\textit{Instructors: \hmwkClassInstructor}}
   %\vspace{-1.5cm}
}

\author{
    \textit{Team: \textbf{\hmwkTeam}}
}
\date{}

\begin{document}

%\pagenumbering{gobble}
\maketitle
\thispagestyle{empty}
\thispagestyle{fancy}
% \pagebreak
\pagenumbering{arabic}

\lettrine[findent=1pt]{\fbox{深}}{ }度強化學習(Deep reinforcement learning)是利用既有的強化學習演算法,結合近年來表現很好的深度學習,所形成的一類機器學習模型,目前在許多領域,例如電腦對局、機器人學、以及機器玩遊戲等,已有很好甚至超越人類的成果。不過,根據先前作業的經驗,傳統的深度強化學習模型在訓練期間頗為耗時,而且就算使用相同環境訓練相同模型,其訓練的成效有時也好壞不一。為了解決這樣的問題,(T. Hester et al., 2017)\cite{DBLP:journals/corr/HesterVPLSPSDOA17} 提出了 DQfD (Deep Q-Learning from Demonstrations)模型,並說明該模型可以僅僅藉由少量的遊戲演示資料來加速模型的訓練。本報告將以數種Atari遊戲作為訓練環境,比較DQfD以及DQN(以及其變形)在這些遊戲中的學習狀態以及成果,並分析導致這些模型結果上差異的因素。

\section{深度強化學習簡介}
在強化學習中,一個模型裡存有一環境(environment)以及主體(agent),而主體的行為模式可視為一馬可夫決策過程(Markov desicion process,MDP)。MDP可以以一個五元組$(S,A,R(\cdot,\cdot),T(\cdot, \cdot, \cdot),\gamma)$表示。$S$代表著所有狀態(state)的集合;$A$為所有可能行動(action)的集合;$R(s,a)$為一獎勵函數(reward function)(即給定目前狀態$s$以及目前行動$a$,其獎勵為多少);$T(s,a,s')=P(s'|s,a)$為一轉換函數(transition function),並服從某一機率分佈;$\gamma$則為折減率(discount factor)。而主體在這個環境中,會根據某個策略函數$\pi(s)$來行動。\par
在強化學習中,最常見的模型根據主體的性質主要分為基於策略(policy-based)以及基於價值(value-based)的模型:前者主要為直接尋找一個策略函數$\pi$使得其盡量接近最佳策略(即$\pi \rightarrow \pi^*$);而後者則是給定一個價值函數($Q^\pi (s,a)$),該函數的目的即是估計出在目前狀態$s$下,若採取某行動$a$所能帶來的預期價值(expected value),而我們會希望該函數可以準確估計出各種狀況下的預期價值(即$Q^\pi (s,a) \rightarrow Q^* (s,a)$),在這情況下,主體的最佳策略即為$\pi^*(s) = \arg \underset{a \in A}\max\ Q^*(s,a)$。\par
Deep Q-Learning(DQN,直譯為「深度Q學習」)是一種常見的基於價值的深度強化學習模型。它的最佳價值函數可以以下面方程式表示:
\[Q^*(s,a) = \mathbb{E}\left[R(s,a) + \gamma \sum_{s'}P(s'|s,a) \underset{a'}\max\ Q^*(s',a')\right]\]
在DQN裡,我們會用一個深度學習網路來代表$Q^\pi$,而我們希望最後$Q^\pi$盡可能的接近$Q^*$。在實務上,我們會用平均平方錯誤(mean squared error, MSE)作為訓練時的誤差函數,且為了穩定訓練,還會固定目標項之參數,如下所示:
\[\mathcal{L}(w) = \mathbb{E}\left[\left(\underbrace{R(s,a) + \gamma \underset{a'}\max\ Q(s',a', w^-)}_{\text{target, update slowly}} - \underbrace{Q(s,a,w)}_{\text{online, update quickly}}\right)^2\right]\]
\par
其他常見的DQN模型還有:Double Q-Learning(DDQN)(H. van Hasselt et al., 2015)\cite{DBLP:journals/corr/HasseltGS15}以及Dueling Network (Z. Wang et al., 2015)\cite{DBLP:journals/corr/WangFL15}等。前者認為,利用目標網路(target network)選取未來預期最大價值的行動,容易有過度估計上的誤差(upward bias)。因此,該模型修正誤差函數成如下:
\[\mathcal{L}(w) = \mathbb{E}\left[\left(R(s,a) + \gamma Q(s',\underbrace{\arg \underset{a'}\max\ Q(s',a',w)}_{\text{online network chooses optimal } a'}, w^-) - Q(s,a,w)\right)^2\right]\],而後者則將網路修正成:
\[Q(s,a,w) = \underbrace{V(s,w)}_{\text{\emph{value}, action-independent}} + \underbrace{A(s,a,w)}_{\text{\emph{advantage}, action-dependent}}\]
基本概念為:有些狀態天生較好(無論行動為何,預期價值一定比較高),而有些則否,因此預期價值可以視為是目前狀態天生的預期價值(獨立於行動),加上在該狀態下,不同動作下所能提昇/減少的價值。

\section{DQfD:利用演示資料學習}
DQfD是一種混合了部份監督式學習(supervised learning)要素的強化學習演算法。相較於原始的DQN,DQfD在真正與環境互動訓練之前,會先使用一些事先收集好的演示資料(demonstration data)來進行預訓練(pre-training),之後才真正與環境互動進行強化。以真實的例子來比喻的話,這就有如一名運動員,先接受教練的訓練(專家演示)之後,才到比賽中(環境)累積經驗、強化自己的能力,而非直接在比賽中摸索。


\section{實驗設定與結果分析}

\section{結論}



\renewcommand\refname{參考文獻}
\bibliographystyle{abbrv}
\bibliography{report}

\end{document}

